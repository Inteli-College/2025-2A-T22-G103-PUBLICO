\documentclass[12pt]{article}
\usepackage[utf8]{inputenc}
\usepackage[english]{babel}
\usepackage{geometry}
\usepackage{hyperref}
\geometry{a4paper, margin=1in}

\title{Impact of Digital Signatures on Latency and Non-Repudiation in High-Frequency Banking Communications}
\author{Esther Hikari \\ Institute of Technology and Leadership}
\date{Not finished yet, 2026}

\begin{document}
\maketitle

\begin{abstract}
This paper investigates the impact of implementing digital signatures on latency and non-repudiation guarantees in high-frequency banking communications. The objective is to establish a methodological baseline for measuring computational costs, latency variation, and legal security gains.
\end{abstract}

\textbf{Keywords:} digital signatures, latency, non-repudiation, banking systems, high-frequency.

\section{Introduction}
% WRITING ORDER: 3rd - Write after having a clear direction from methodology and initial results
% Write when you understand the full scope of what was investigated
Present the context of high-frequency banking communications, reliability requirements, and the role of digital signatures.

\section{Problem Statement and Hypotheses}
% WRITING ORDER: 2nd - Define early, but refine after methodology is clear
% Helps guide the experimental design
Define the central problem and hypotheses about the trade-off between latency and non-repudiation guarantees.

\section{Objectives}
% WRITING ORDER: 2nd - Define alongside problem and hypotheses
% Sets clear goals before starting experiments
Will implement in Sprint 2 to outline specific objectives related to measuring latency impacts and non-repudiation benefits.

\section{Theoretical Foundation}
% WRITING ORDER: 1st - START HERE to build theoretical foundation
% Research and understand key concepts before designing experiments
Will implement in Sprint 2 to synthesize concepts of digital signatures, network latency, and non-repudiation properties.

\section{Methodology}
% WRITING ORDER: 2nd - Design after understanding fundamentals
% Plan experimental approach before implementation
Will implement in Sprint 3 to detail the experimental design, including test scenarios, control variables, and data collection methods.

\section{Experiment Architecture}
% WRITING ORDER: 3rd - Implement after methodology is defined
% Detail actual implementation and setup
Will implement in Sprint 3 to explain the environment, components, test workloads, and key and certificate configurations.

\section{Metrics and Instrumentation}
% WRITING ORDER: 3rd - Define alongside experiment architecture
% Specify what and how you'll measure
Will implement in Sprint 3 to define metrics (p99, p999, throughput, message size) and collection instruments.

\section{Expected Results}
% WRITING ORDER: 4th - Write after running experiments and collecting data
% Present findings objectively
Will implement in Sprint 4 to indicate expected results and possible patterns of latency degradation.

\section{Threats to Validity}
% WRITING ORDER: 4th - Identify after analyzing results
% Critically assess limitations discovered during experimentation
 Will implement in Sprint 4 to describe evaluation limits, biases, and external factors.

\section{Conclusion}
% WRITING ORDER: 5th (LAST) - Write after everything else is complete
% Synthesize findings and provide closure
Will implement in Sprint 5 to recap the value of the study and propose next steps.

\section*{References}
Just a few key references to be expanded upon after the literature review is complete.

\begin{thebibliography}{9}

\bibitem{rsa1978}
Rivest, R. L., Shamir, A., and Adleman, L. (1978). A method for obtaining digital signatures and public-key cryptosystems. \textit{Communications of the ACM}, 21(2), 120-126.

\bibitem{bernstein2012}
Bernstein, D. J., Duif, N., Lange, T., Schwabe, P., and Yang, B. Y. (2012). High-speed high-security signatures. \textit{Journal of Cryptographic Engineering}, 2(2), 77-89.

\bibitem{hasbrouck2013}
Hasbrouck, J., and Saar, G. (2013). Low-latency trading. \textit{Journal of Financial Markets}, 16(4), 646-679.

\bibitem{dean2013}
Dean, J., and Barroso, L. A. (2013). The tail at scale. \textit{Communications of the ACM}, 56(2), 74-80.

\bibitem{zhou2001}
Zhou, J., and Gollmann, D. (2001). Evidence and non-repudiation. \textit{Journal of Network and Computer Applications}, 24(4), 343-371.

\end{thebibliography}

\end{document}
